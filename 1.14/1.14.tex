%SICP 1.14习题
\documentclass{article}
\usepackage{CJK}
\usepackage{amssymb}
\usepackage{tikz}
\usepackage{tikz-qtree}
\begin{document}
\begin{CJK}{UTF8}{gkai}
问:画出一棵树,可以用来说明(count-change 11)的函数调用过程。时间和空间增长的阶的增长情况是怎么样的?

答:首先,可用的面额有(1 5 10 25 50)。调用过程可以用(剩余金额, 选择面额)来表示。 当“剩余面额”为0
的时候,就表示程序调用成功。可以使用的面额不能比之前的小,就是说如果面额:5的钱币用了,之后就不能使用
面额为1的钱币。为了具体的调用过程如下图:
\begin{center}
\begin{tikzpicture}[grow=down]
\tikzset {level distance=30pt }
\tikzset {sibling distance=5pt}
\tikzset{frontier/.style={distance from root=330pt}}
\tikzset{edge from parent/.style=
{draw,
edge from parent path={(\tikzparentnode.south)
-- +(0,-6pt)
-| (\tikzchildnode)}}}

\Tree [ .(11-X) 
               [ .(10-1)
                         [ .(9-1) 
                                  [ .(8-1) 
                                          [ .(7-1) 
                                                   [ .(6-1)
                                                           [ .(5-1)
                                                                    [ .(4-1) [ .(3-1) [ .(2-1) [ .(1-1) (0-1)$\checkmark$ ]]]]
                                                                       (0-5)$\checkmark$ ]
                                                              (1-5)$\times$ ]
                                                      (2-5)$\times$ ]
                                           (3-5)$\times$ ]
                                   (4-5)$\times$ ]  
                         [ .(5-5) (0-5)$\checkmark$ ] 
                           (0-10)$\checkmark$ ] 
                [ .(6-5)  [ .(1-5)$\times$ ]] 
                [ .(1-10)$\times$ ]]
\end{tikzpicture}
\end{center}

具体的来说,调用的空间占用情况为11单元, 时间占用情况为23单元。答案是(1 1 1 1 1 1 1 1 1 1 1) (1 1 1 1 1 1 5) (1 5 5) (1 10)
\end{CJK}
\end{document}